\documentclass[12pt,pdftex,a4paper]{article}
\usepackage[ngerman]{babel}
\usepackage[utf8]{inputenc}
\usepackage{amsmath}
\usepackage{amssymb}
\usepackage[pdftex]{graphicx}
\begin{document}
\title{Bachelor-Arbeit\\Exposé}
\author{Raphael Br\"osamle}
\date{Institut f\"ur Informationssicherheit\\
Universit\"at Stuttgart\\
Betreuer: Dr. Kay Schweiger}
\maketitle
%%%%%%%%%%%%%%%%%%%%%%%%%%%%%%%%%%%%%%%%%%%%%%%%
Das Internet ist ein immer gr\"o{\ss}er werdender Bestandteil des Alltags. Ob zuhause oder unterwegs, eine schnelle und sichere Internetverbindung ist f\"ur viele Menschen wichtig. Daf\"ur ist unter anderem das mobile Netz ausschlaggebend. Deshalb wird viel in den Ausbau des Netzes und in neue Technologien investiert. Vor allem die kommende 5G-Technologie spielt dabei eine gro{\ss}e Rolle, da sie schnelles und wesentlich sichereres Internet verspricht. Aber ist die Internetverbindung durch die 5G-Technologie tats\"achlich sicher?\\
Dies gilt es in dieser Bachelor-Arbeit herauszufinden. Daf\"ur wird das 5G-AKA-Protokoll unter die Lupe genommen. Im Wesentlichen wird folgendes betrachtet: 
\begin{itemize}
\item Es wird das 5G-AKA Protokoll beschrieben. Daf\"ur werden die Ziele des Protokolls genannt und es wird dargestellt, wie das 5G-AKA-Protokoll funktioniert.
\item Es werden die Unterschiede zwischen den Authentifizierungsmethoden von 5G und 4G aufgezeigt. Dabei werden die Ziele des 4G EPS-AKA-Protokolls genannt und die Verbesserungen des 5G-AKA-Protokolls im Vergleich zum EPS-AKA-Protokoll erkl\"art.
\item Es wird eine bereits bekannte Sicherheitsl\"ucke in dem 5G-AKA-Protokoll genannt und beschrieben. Dabei handelt es sich um die Sicherheitsl\"ucke, die in \textit{Security vulnerability in 5G-AKA draft}\cite{dehnel2018security} von \textit{Martin Dehnel-Wild and Cas Cremers} erkl\"art wurde. Des Weiteren wird noch erkl\"art welche Auswirkungen diese L\"ucke auf die Sicherheit des Protokolls hat.
\item Diese Sicherheitslücke\cite{dehnel2018security} wird implementiert, um dessen Praktikabilit\"at herauszufinden. Die Authentifizierung soll dabei mit Hilfe der Sicherheitsl\"ucke umgangen werden.
\end{itemize}

\bibliographystyle{plain}	
%	\section*{Literaturhinweise}
\bibliography{references}

\end{document}

