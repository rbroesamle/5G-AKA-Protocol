\chapter{Beschreibung einer Sicherheitsl\"ucke}
\label{chap:4}

Noch bevor eine endgültige Fassung des 5G-AKA Protokolls feststand haben sich einige Forscher bereits mit unterschiedlichen Sicherheitsaspekten des Protokolls auseinandergesetzt und teilweise sogar mögliche Sicherheitslücken entdeckt.
\textit{Martin Dehnel-Wild} und \textit{Cas Cremers} haben in \textit{Security vulnerability in 5G-AKA draft} solch eine Sicherheitslücke vorgestellt. %Security vulnerability in 5G-AKA draft
Ihre Befunde beziehen sich auf die Spezifikation TS 33.501 V0.7.0 des 3GPP. %3GPP TS 33.501 V0.7.0
mittlerweile hat das 3GPP bereits neuere Versionen der Spezifikation veröffentlicht, darunter auch die Version V15.34.1, auf der die Beschreibung aus \cref{chap:2} beruht. %3GPP TS 33.501 V15.34.1
Da sich zwischen den Versionen V0.7.0 und V15.34.1 vor allem die Benennung einiger Komponenten verändert hat, wird zur Verbesserung der Übersichtlichkeit im weiteren Verlauf die Sicherheitslücke im Kontext der Spezifikation TS 33.501 V15.34.1 beschrieben.

\section{Beschreibung der Sicherheitslücke}

\section{Auswirkungen der Sicherheitslücke}