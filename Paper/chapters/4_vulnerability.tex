\chapter{Beschreibung einer Sicherheitsl\"ucke}
\label{chap:4}

Noch bevor eine endgültige Fassung des 5G-AKA Protokolls feststand haben sich einige Forscher bereits mit unterschiedlichen Sicherheitsaspekten des Protokolls auseinandergesetzt und teilweise sogar mögliche Sicherheitslücken entdeckt.
\textit{Martin Dehnel-Wild} und \textit{Cas Cremers} haben in \textit{Security vulnerability in 5G-AKA draft} solch eine Sicherheitslücke vorgestellt. %Security vulnerability in 5G-AKA draft
Ihre Befunde beziehen sich auf die Spezifikation TS 33.501 V0.7.0 des 3GPP. %3GPP TS 33.501 V0.7.0
mittlerweile hat das 3GPP bereits neuere Versionen der Spezifikation veröffentlicht, darunter auch die Version V15.34.1, auf der die Beschreibung aus \cref{chap:2} beruht. %3GPP TS 33.501 V15.34.1
Da sich zwischen den Versionen V0.7.0 und V15.34.1 vor allem die Benennung einiger Komponenten verändert hat, wird zur Verbesserung der Übersichtlichkeit im weiteren Verlauf die Sicherheitslücke im Kontext der Spezifikation TS 33.501 V15.34.1 beschrieben.


\section{Beschreibung der Sicherheitslücke}

Die in \textit{Security vulnerability in 5G-AKA draft} vorgestellte Sicherheitslücke ermöglicht es einem Angreifer sich als ein anderer Benutzer auszugeben. %Security vulnerability in 5G-AKA draft
Damit dieser Angriff erfolgreich ausgeführt werden kann müssen einige Vorbereitungen getroffen werden.

\subsection{Vorbereitungen}

\begin{enumerate}
\item Der Angreifer muss an den \gls{suci} des Benutzers kommen für den er sich ausgeben möchte.
Dafür muss er die \textit{N1 message} des Benutzers mithören und aufzeichnen.

\item Hat der Angreifer die \textit{N1 message} des Benutzers aufgezeichnet, dann bringt er in Erfahrung zu welchem \textit{Home Network} der \gls{suci} gehört und kauft sich ein legitimes \gls{usim} des selben \textit{Home Network}s.

\item Ist der Angreifer im Besitz des legitimen \gls{usim} so kompromittiert er dieses und extrahiert daraus den Langzeitschlüssel \gls{k}.

\end{enumerate}

\subsection{Der Hauptteil des Angriffs}




\section{Auswirkungen der Sicherheitslücke}
