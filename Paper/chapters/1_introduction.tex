\chapter{Einleitung}
\label{chap:1}

In diesem Kapitel steht die Einleitung zu dieser Arbeit.
Sie soll nur als Beispiel dienen und hat nichts mit dem Buch \cite{WSPA} zu tun.
Nun viel Erfolg bei der Arbeit!

Bei \LaTeX\ werden Absätze durch freie Zeilen angegeben.
Da die Arbeit über ein Versionskontrollsystem versioniert wird, ist es sinnvoll, pro \emph{Satz} eine neue Zeile im \texttt{.tex}-Dokument anzufangen.
So kann einfacher ein Vergleich von Versionsständen vorgenommen werden.

Die Arbeit ist in folgender Weise gegliedert:
In \cref{chap:k2} werden die Grundlagen dieser Arbeit beschrieben.
Schließlich fasst \cref{chap:zusfas} die Ergebnisse der Arbeit zusammen und stellt Anknüpfungspunkte vor.
