\chapter{Einleitung}
\label{chap:1}

Die Mobilfunknetze haben sich in den letzten Jahren stark weiterentwickelt. %TODO Filler here
Vor allem die 5G Technologie soll deutliche Verbesserungen mit sich bringen. 
Dabei soll neben deutlich schnellerem Internet auch, im Gegensatz zu vorherigen Generationen, die Sicherheit sp\"urbar verbessert werden.
Um dies zu gew\"ahrleisten wurden neue Standards entwickelt. 
Bei diesen Standards handelt es sich um neue Protokolle, wie das 5G-AKA-Protokoll zum sicheren Austausch eines gemeinsamen Schl\"ussels und viele weitere Protokolle.

Was Sicherheit angeht ist vor Allem der sichere Austausch eines gemeinsamen Schl\"ussels von gro{\ss}er Bedeutung. 
In den fr\"uheren Generationen der Mobilfunktechnologie (2G, 3G und 4G) wurde die Sicherheit der Protokolle bereits weitreichend untersucht und auch f\"ur die Protokolle des 5G Standards wurden bereits einige Untersuchungen angestellt. %https://www.cablelabs.com/insights/a-comparative-introduction-to-4g-and-5g-authentication


\section{4G-Standard}


\section{5G-Standard}


\section{Ziel der Arbeit}
Ziel dieser Arbeit ist es einen \"Uberblick \"uber das 5G-AKA-Protokoll zu erlangen. Daf\"ur wird das Protokoll beschrieben und die Unterschiede und \"Ahnlichkeiten zu dessen Vorg\"anger, dem 4G EPS-AKA-Protokoll, aufgezeigt. Des Weiteren wird auf die Sicherheit des Protokolls eingegangen, indem eine Sicherheitsl\"ucke dieses Protokolls beschrieben und implementiert. %https://www.cs.ox.ac.uk/5G-analysis/5G-AKA-draft-vulnerability.pdf


\section{Aufbau der Arbeit}
In \cref{chap:2} wird das 5G-AKA-Protokoll beschrieben. Dabei wird genauer auf dessen Funktionsweise eingegangen, sowie das Ziel des Protokolls erl\"autert. In \cref{chap:3} wird das 5G-AKA-Protokoll mit dem 4G EPS-AKA-Protokoll des 4G Standards verglichen. 
Daf\"ur wird das 4G EPS-AKA-Protokoll kurz geschrieben und dessen Ziele erl\"autert. Des Weiteren werden die Unterschiede der Protokolle hervorgehoben und die Verbesserungen des neuen Protokolls, gegen\"uber dem Alten, aufgezeigt.
\cref{chap:4} gibt einen \"Uberblick \"uber eine Sicherheitsl\"ucke des 5G-AKA-Protokolls. Daf\"ur wird die Sicherheitsl\"ucke beschrieben und deren Auswirkungen aufgezeigt. In \cref{chap:5} wird die Implementierung der beschriebenen Sicherheitsl\"ucke dokumentiert und in \cref{chap:6} ist eine Zusammenfassung dieser Arbeit und ein Ausblick zu finden.
