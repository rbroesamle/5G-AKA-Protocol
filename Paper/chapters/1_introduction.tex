\chapter{Einleitung}
\label{chap:1}

Die Mobilfunknetze haben sich in den letzten Jahren stark weiterentwickelt. %TODO Filler here
Vor allem die 5G Technologie soll deutliche Verbesserungen mit sich bringen. 
Dabei soll neben deutlich schnellerem Internet auch, im Gegensatz zu vorherigen Generationen, die Sicherheit spürbar verbessert werden.
Um dies zu gewährleisten wurden neue Standards entwickelt. 
Bei diesen Standards handelt es sich um neue Protokolle, wie das 5G-AKA-Protokoll zum sicheren Austausch eines gemeinsamen Schlüssels und viele weitere Protokolle.

Was Sicherheit angeht ist vor Allem der sichere Austausch eines gemeinsamen Schlüssels von großer Bedeutung. 
In den früheren Generationen der Mobilfunktechnologie (2G, 3G und 4G) wurde die Sicherheit der Protokolle bereits weitreichend untersucht und auch für die Protokolle des 5G Standards wurden bereits einige Untersuchungen angestellt. %https://www.cablelabs.com/insights/a-comparative-introduction-to-4g-and-5g-authentication


\section{4G-Standard}
Der 4G-Standard ist die vierte Generation der Breitband-Mobilfunknetz-Technologie. %https://en.wikipedia.org/wiki/4G
Sein Vorgänger ist der 3G-Standard. 
Der 4G-Standard verspricht, gegenüber dem 3G-Standard, deutlich schnellere Übertragungsraten von bis zu 100MBit/s und in manchen Situationen auch von bis zu 1GBit/s. %https://en.wikipedia.org/wiki/4G
Häufig wird 4G mit LTE (Long-Term Evolution) gleichgesetzt. %https://www.digitaltrends.com/mobile/4g-vs-lte/
Dies ist jedoch nicht ganz korrekt, da die Übertragungsraten des LTE-Standards die vom 4G-Standard geforderten Übertragungsraten leicht unterschreitet. 
Durch diese nur knappe Unterschreitung wird LTE von manchen Forschern auch als 3.9G bezeichnet. %https://citeseerx.ist.psu.edu/viewdoc/download?doi=10.1.1.642.2509&rep=rep1&type=pdf#page=266

Mit LTE-A (LTE-Advanced) wurde eine Verbesserung des LTE-Standards entwickelt, der offiziell die Vorraussetzungen des 4G-Standards erfüllt. %https://en.wikipedia.org/wiki/LTE_Advanced
Diese Verbesserung wurde von dem 3GPP (3rd Generation Partnership Project) eingebracht, der auch zu der Entwicklung des 5G-Standards maßgeblich beigetragen hat. %https://en.wikipedia.org/wiki/3GPP


\section{5G-Standard}
Der 5G-Standard ist die f\"unfte Generation der Breitband-Mobilfunk-Technologie. %https://en.wikipedia.org/wiki/5G
Er ist der Nachfolger des 4G-Standards und der bisher neueste Standard auf dem Markt.
Er wurde von dem 3GPP definiert und soll gegen Ende 2019 von mindestens 10\% aller Mubilfunktelefone unterst\"utzt werden. %https://en.wikipedia.org/wiki/5G

Der 3GPP hat auch den 5G NR (5G New Radio) entwickelt, der als RAT (Radio Access Technology) f\"ur mobile Netze gedacht ist. %https://ieeexplore.ieee.org/abstract/document/8258595
Dieser erf\"ullt nach Definition die Vorraussetzungen des 5G-Standards. %https://en.wikipedia.org/wiki/5G
Der 5G NR wird in zwei Modi ausgerollt werden. 
Der erste Modus ist der NSA mode (Non-Standalone Mode), der von dem bisherigen LTE Netzwerk abh\"angig ist. %https://en.wikipedia.org/wiki/5G_NR
Dieser Modus wird dann im Laufe der Zeit von dem SA mode (Standalone Mode) abgel\"o{\ss}t, der unabh\"angig von bisherigen Netzwerken funktioniert. %https://en.wikipedia.org/wiki/5G_NR


\section{Ziel der Arbeit}
Ziel dieser Arbeit ist es einen \"Uberblick \"uber das 5G-AKA-Protokoll zu erlangen. Daf\"ur wird das Protokoll beschrieben und die Unterschiede und \"Ahnlichkeiten zu dessen Vorg\"anger, dem 4G EPS-AKA-Protokoll, aufgezeigt. Des Weiteren wird auf die Sicherheit des Protokolls eingegangen, indem eine Sicherheitsl\"ucke dieses Protokolls beschrieben und implementiert. %https://www.cs.ox.ac.uk/5G-analysis/5G-AKA-draft-vulnerability.pdf


\section{Aufbau der Arbeit}
In \cref{chap:2} wird das 5G-AKA-Protokoll beschrieben. 
Dabei wird genauer auf dessen Funktionsweise eingegangen, sowie das Ziel des Protokolls erl\"autert. I
n \cref{chap:3} wird das 5G-AKA-Protokoll mit dem 4G EPS-AKA-Protokoll des 4G Standards verglichen. 
Daf\"ur wird das 4G EPS-AKA-Protokoll kurz geschrieben und dessen Ziele erl\"autert. 
Des Weiteren werden die Unterschiede der Protokolle hervorgehoben und die Verbesserungen des neuen Protokolls, gegen\"uber dem Alten, aufgezeigt.
\cref{chap:4} gibt einen \"Uberblick \"uber eine Sicherheitsl\"ucke des 5G-AKA-Protokolls. 
Daf\"ur wird die Sicherheitsl\"ucke beschrieben und deren Auswirkungen aufgezeigt. 
In \cref{chap:5} wird die Implementierung der beschriebenen Sicherheitsl\"ucke dokumentiert und in \cref{chap:6} ist eine Zusammenfassung dieser Arbeit und ein Ausblick zu finden.
