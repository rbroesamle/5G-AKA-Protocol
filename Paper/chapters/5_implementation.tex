\chapter{Implementierung der Sicherheitsl\"ucke}
\label{chap:5}

Die in \cref{chap:4} beschriebene Sicherheitslücke und das 5G-AKA Protokoll wurde nun implementiert.
Die Implementierung ist dieser Arbeit beigefügt und kann auf Github unter folgendem Link eingesehen werden: \url{}//TODO
In diesem Kapitel wird sich auf den Commit \#ID gezogen.//TODO

Es wurde das 5G-AKA Protokoll aus der Spezifikation TS 33.501 des 3GPP mit der Version V15.34.1 implementiert.
Die Sicherheitslücke wurde darauf aufbauend implementiert.


\section{Beschreibung des Aufbaus}
Für die Implementierung wurde die Programmiersprache Java gewählt.

Der Einstiegspunkt des Programms ist die \lstinline{Implementation.App} Klasse. Von dort lässt sich das Programm mit der \lstinline{main} Methode aufrufen.

In dem \lstinline{Implementation.structure} Package wird die grundlegende Struktur für Entitäten und Nachrichten definiert.
Alle Entitäten haben die \lstinline{Implementation.structure.Entity} Klasse als Superklasse und alle Nachrichten haben das \lstinline{Implementation.structure.Message} Interface implementiert.

Das \lstinline{Implementation.helper} Package beinhaltet Helferklassen.
Diese Helferklassen werden verwendet um die Übersichtlichkeit in den anderen Packages zu verbessern.
Sie beinhalten nur statische Methoden.

In dem \lstinline{Implementation.protocol} Package ist die tatsächliche Implementierung des Protokolls zu finden.
Es ist unterteilt in die Packages \lstinline{Implementation.protocol.additional}, \lstinline{Implementation.protocol.data}, \lstinline{Implementation.protocol.entities} und \lstinline{Implementation.protocol.messages}.

Das \lstinline{Implementation.protocol.additional} Package beinhaltet hauptsächlich Klassen die kryptographischen Berechnungen ausführen, wie die \gls{kdf}.
Alle Klassen in diesem Package beinhalten nur statische Methoden.

Das \lstinline{Implementation.protocol.data} Package beinhaltet Klassen, die mehrere verschiedene Parameter in sich zusammenfassen, wie z.B. der \gls{5g-he-av}.

In dem \lstinline{Implementation.protocol.entities} Package ist die Implementierung der vier Entitäten \gls{ue}, \gls{seaf}, \gls{ausf} und \gls{udm} zu finden, so wie eine \lstinline{EvilUE} Klasse, die den Angreifer darstellen soll.
In diesen Klassen ist das Versenden und Antworten auf Nachrichten implementiert.
Dies geschieht über die \lstinline{Implementation.structure.Entity} Superklasse.

In dem \lstinline{Implementation.protocol.messages} Package sind alle Nachrichten zu finden, die die Entitäten untereinander verschicken.
Jede Nachricht hat eine eigene Klasse, die das \lstinline{Implementation.structure.Message} Interface implementiert.

\subsection{Implementierung der Race Condition}


\subsection{Unterschiede zum Protokoll}
Ziel dieser Implementierung ist es die Praktikabilität der in \cref{chap:4} beschriebenen Sicherheitslücke herauszufinden.
In der Implementierung wurde daher einige Teile der Protokolls, die für die Sicherheitslücke nicht relevant sind, verändert oder weggelassen.
Nachfolgend ist eine Liste aller Abweichungen vom Protokoll aufgeführt.
Alle Abweichungen sind im Code mit dem Kommentar \lstinline{//MARK: Deviation #} versehen.
Das \lstinline{#} entspricht der Nummer der Abweichung.

\begin{enumerate}
\item //TODO: Continue
//TODO: Den Codekommentar MARK: Deviation \# setzen
\end{enumerate}

\section{Ausführen des Programms}
//TODO: Continue



\section{Übersicht der Komponenten}
