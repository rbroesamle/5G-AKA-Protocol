\chapter{Beschreibung des 5G-AKA Protokolls}
\label{chap:2}

Das 5G-AKA-Protokoll ist eines von mehreren Protokollen des 5G-Standards.
Vor Allem das EAP-AKA-Protokoll und das 5G-AKA-Protokoll sind f\"ur den sicheren Austausch eines Schl\"ussels zust\"andig. 
Bei beiden Protokollen ist der erste Teil der Gleiche bis bei dem Provider eines der Protokolle ausgew\"ahlt wird. %3GPP TS 33.501 V15.34.1 Page 40
Bei beiden Protokollen ist auch der Authentifizierungsvorgang sehr \"ahnlich, wobei jedoch der das 5G-AKA-Protokoll das EAP-AKA-Protokoll darum erweitert, dass es dem Provider, ''home network'' genannt, eine erfolgreiche Authentifizierung nachweist. %3GPP TS 33.501 V15.34.1 Page 43
Daher wird in dieser Ausarbeitung haupts\"achlich auf das 5G-AKA-Protokoll eingegangen.

Beide Protokolle werden gr\"o{\ss}tenteils in der Spezifikation TS 33.501 des 3GPP beschrieben.%3GPP TS 33.501 V15.34.1
Dabei wird die zurzeit neueste Version V15.34.1 herangezogen.

\section{Ziele}

TODO

\section{Das Protokoll}

\subsection{Entities}

In dem Protokoll werden vier grundlegende Entit\"aten beschrieben.
Diese sind das \gls{ue}, die \gls{seaf}, die \gls{ausf} und das \gls{udm}.

\glsreset{ue}
\subsubsection{\gls{ue}}

Das \gls{ue} befindet sich auf dem Smartphone des Nutzers.
Es l\"asst sich in das \gls{me} und das \gls{usim} unterteilen.
In dem \gls{usim} werden f\"ur die Authentifizierung ben\"otigte Schl\"ussel und Benutzerkennungen gespeichert, die f\"ur die eindeutige Identifizierung und Authentifizierung des \gls{ue} ben\"otigt werden. %3GPP TS 33.501 V15.34.1 Page 24
In dem \gls{me} werden die Schl\"ussel, die f\"ur die Kommunikation nach dem 5G-AKA-Protokoll ben\"otigt werden berechnet.

\glsreset{seaf}
\subsubsection{\gls{seaf}}

Die \gls{seaf} ist Teil des mit dem \gls{ue} kommunizierenden Netzwerks, auch \textit{''Serving Network''‚} genannt.
Sie kommuniziert mit dem \gls{ue} und mit dem korrekten Provider des Benutzers.
Nach erfolgreicher Authentifizierung haben das \gls{ue} und die \gls{seaf} beide einen gemeinsamen Schl\"ussel.%3GPP TS 33.501 V15.34.1 Page 37

\glsreset{ausf}
\subsubsection{\gls{ausf}}

Die \gls{ausf} ist Teil des Providers, bei dem sich der Benutzer die hier verwendete SIM-Karte gekauft hat, auch \textit{''Home Environment''} genannt.
Sie kommuniziert mit der \gls{seaf} und validiert die Antwort des \gls{ue} f\"ur die \gls{seaf}.

\glsreset{udm}
\subsubsection{\gls{udm}}

Das \gls{udm} ist, wie auch die \gls{ausf} Teil des \textit{''Home Environment''}.
Grunds\"atzlich funktioniert es jedoch nicht ohne die \gls{arpf} und die \gls{sidf}.
Daher wird im Weiteren Verlauf oft die Abk\"urzung \gls{udm}/\gls{arpf}/\gls{sidf} verwendet.

Das \gls{udm} und die \gls{arpf} generieren Authentifizierungsvektoren aus den Schl\"usseln, die sie mit dem \gls{usim} teilen.
Die \gls{sidf} gerechnet die Benutzerkennung f\"ur den \gls{udm} und die \gls{arpf} aus einer verschl\"usselten Benutzerkennung, die sie von dem \gls{ausf} erh\"alt.


\subsection{Wichtige Definitionen}

\glsreset{5g-av}
\subsubsection{\gls{5g-av}}

\glsreset{5g-guti}
\subsubsection{\gls{5g-guti}}

\glsreset{5g-he-av}
\subsubsection{\gls{5g-he-av}}

\glsreset{5g-se-av}
\subsubsection{\gls{5g-se-av}}

\glsreset{ak}
\subsubsection{\gls{ak}}

\glsreset{amf}
\subsubsection{\gls{amf}}

\glsreset{autn}
\subsubsection{\gls{autn}}

\glsreset{ck}
\subsubsection{\gls{ck}}

\glsreset{hres*}
\subsubsection{\gls{hres*}}

\glsreset{hxres*}
\subsubsection{\gls{hxres*}}

\glsreset{ik}
\subsubsection{\gls{ik}}

\glsreset{k-ausf}
\subsubsection{\gls{k-ausf}}

\glsreset{k-seaf}
\subsubsection{\gls{k-seaf}}

\glsreset{rand}
\subsubsection{\gls{rand}}

\glsreset{res}
\subsubsection{\gls{res}}

\glsreset{res*}
\subsubsection{\gls{res*}}

\glsreset{sn-name}
\subsubsection{\gls{sn-name}}

\glsreset{suci}
\subsubsection{\gls{suci}}

\glsreset{supi}
\subsubsection{\gls{supi}}

\glsreset{xres}
\subsubsection{\gls{xres}}

\glsreset{xres*}
\subsubsection{\gls{xres*}}




\subsection{Weitere Begriffe}

\glsreset{abba}
\subsubsection{\gls{abba}}

\glsreset{ng-ksi}
\subsubsection{\gls{ng-ksi}}
















