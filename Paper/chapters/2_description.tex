\chapter{Beschreibung des 5G-AKA Protokolls}
\label{chap:2}

Das 5G-AKA-Protokoll ist eines von mehreren Protokollen des 5G-Standards.
Vor Allem das EAP-AKA-Protokoll und das 5G-AKA-Protokoll sind f\"ur den sicheren Austausch eines Schl\"ussels zust\"andig. 
Bei beiden Protokollen ist der erste Teil der Gleiche bis bei dem Provider eines der Protokolle ausgew\"ahlt wird. %3GPP TS 33.501 V15.34.1 Page 40
Bei beiden Protokollen ist auch der Authentifizierungsvorgang sehr \"ahnlich, wobei jedoch der das 5G-AKA-Protokoll das EAP-AKA-Protokoll darum erweitert, dass des dem Provider, ''home network'' genannt, eine erfolgreiche Authentifizierung nachweist. %3GPP TS 33.501 V15.34.1 Page 43
Daher wird in dieser Ausarbeitung haupts\"achlich auf das 5G-AKA-Protokoll eingegangen.

Beide Protokolle werden gr\"o{\ss}tenteils in der Spezifikation TS 33.501 des 3GPP beschrieben.
Dabei wird die zurzeit neueste Version V15.34.1 herangezogen.

\section{Das Protokoll}

Hier wird der Hauptteil stehen.
Falls mehrere Kapitel gewünscht, entweder mehrmals \texttt{\textbackslash{}chapter} benutzen oder pro Kapitel eine eigene Datei anlegen und \texttt{ausarbeitung.tex} anpassen.

LaTeX-Hinweise stehen in \cref{chap:latextipps}.
